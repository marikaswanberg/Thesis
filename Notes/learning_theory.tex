\documentclass[12pt]{article}

\usepackage{amsmath}
\usepackage{amssymb}
\usepackage{mathtools}
\usepackage{color}
\usepackage[total={6in,8in}]{geometry}
\usepackage{amsthm}
\usepackage{mathrsfs}

\usepackage{enumitem}
\usepackage{centernot}

\newcommand{\N}{\mathbb{N}}
\newcommand{\Z}{\mathbb{Z}}
\newcommand{\Q}{\mathbb{Q}}
\newcommand{\R}{\mathbb{R}}
\newcommand{\C}{\mathbb{C}}
\newcommand{\sn}{\mathfrak{S}}
\newcommand{\ve}{\varepsilon}
\setlength{\parindent}{0cm}


\author{Marika Swanberg}
\title{Concept Learning}
\date{}
\begin{document}
\maketitle
Mostly taken from \textit{Improved Bounds on Quantum Learning Algorithms} by A.Atici and R. Servedio.

\section{Preliminaries}
A \textit{concept} $c$ over $\{0,1\}^n$ is a Boolean function $c:\{0,1\}^n \rightarrow \{0,1\}$. Equivalently, we may view a concept as a subset of $\{0,1\}^n$ defined by $\{x\in \{0,1\}^n : c(x) = 1\}$. A \textit{concept class} $\mathcal{C} = \cup_{n\geq 1}C_n$ is a set of concepts where $C_n$ consists of those concepts in $\mathcal{C}$ whose domain is $\{0,1\}^n$.

\bigskip

\textbf{Example 1.} Suppose the concept $c$ is all even (binary) numbers in $\{0,1\}^n$. Then
\[
c(x) =
  \begin{cases}
                                   0 & \text{if $x_n = 1$} \\
                                   1 & \text{if $x_n = 0$}
  \end{cases}
\]
Where $x_n$ is the last bit of $x$.

\section{Application to Coding Theory}
We can view decoding as a concept learning problem. Given a noisy codeword $x$, we would like to know which codeword $x'$ it corresponds to. For example, let $n$ be the number of codewords in $\mathscr{R}(r,m)$. Then, 

\[
c(x) =
  \begin{cases}
                                   1 & \text{if $x$ decodes to $c$} \\
                                   0 & \text{else }
  \end{cases}
\]
Not sure about this.....
\end{document}